\documentclass[a4paper, 10pt]{article}
\topmargin-1.5cm

\usepackage{fancyhdr}
\usepackage{pagecounting}
\usepackage[dvips]{color}

% Color Information from - http://www-h.eng.cam.ac.uk/help/tpl/textprocessing/latex_advanced/node13.html

% NEW COMMAND
% marginsize{left}{right}{top}{bottom}:
%\marginsize{3cm}{2cm}{1cm}{1cm}
%\marginsize{0.85in}{0.85in}{0.625in}{0.625in}

%\advance\oddsidemargin-0.85in
%\advance\evensidemargin-0.85in
%\textheight9.1in
%\textwidth6.75in
\newcommand\bb[1]{\mbox{\em #1}}
\def\baselinestretch{1.25}
%\pagestyle{empty}
\newcommand{\hsp}{\hspace*{\parindent}}
\definecolor{gray}{rgb}{0.4,0.4,0.4}

\newcommand{\authorname}{Kathryn~D.~Huff }
\newcommand{\authoremail}{katyhuff@gmail.com}
\newcommand{\authorsite}{katyhuff.github.com}

\begin{document}
\pagestyle{fancy}
%\pagenumbering{gobble}
%\fancyhead[location]{text} 
% Leave Left and Right Header empty.
%\lhead{}
%\rhead{}
%\rhead{\thepage}
\lhead{\textcolor{gray}{\it \authorname}}
\rhead{\textcolor{gray}{\thepage/\totalpages{}}}
\renewcommand{\headrulewidth}{0pt} 
\renewcommand{\footrulewidth}{0pt} 
\fancyfoot[C]{\footnotesize \textcolor{gray}{\authorsite}} 

\begin{center}
{\LARGE \bf Research Project}\\
\vspace*{0.1cm}
{\normalsize \authorname (\authoremail)}
\end{center}
%\vspace*{0.2cm}

%\begin{document}
%\centerline {\Large \bf Research Statement for \authorname}
%\vspace{0.5cm}

% Write about research interests...
%\footnotemark
%\footnotetext{Check This}


% Introduction
Improving the safety and sustainability of nuclear power requires improved 
nuclear reactor designs, fuel cycle strategies, and waste disposal concepts.  
These systems are sufficiently complex that modern software methods and 
high performance computing resources are essential to understanding and 
improving them. The importance of sophisticated scientific computing in 
advancing safe, sustainable nuclear power drives my research direction toward 
the development of computational methods and simulation tools for nuclear 
science and  reactor physics across many scales, with special focus on the 
safety and security of advanced reactor designs and fuel cycles.

\section*{\textcolor{gray}{\it Future Research}}
%correct a dearth of accuracy, transparency, and access in the current landscape of computational tools for nuclear engineering. 
In the near term, I intend to focus on the development computational methods and 
toolkits for fuel cycle sustainability and reactor safety evaluation.
Novel solutions to challenges in nuclear energy such as nuclear reactor accident 
response, nuclear fuel cycle strategy, and waste management will arise from the  
development of application-driven numerical simulation methods and incorporation 
of sophisticated computational tools. To this end, my work will range from 
individual contributions to existing toolkits to comprehensive solution 
libraries for specific nuclear applications (e.g., a suite of activation 
analysis methods for sub-critical systems). Open source projects such as 
Cyclus \cite{huff_cyclus_2011}, MOOSE \cite{gaston_moose:_2009}, and the Python 
for Nuclear Engineering (PyNE) toolkit \cite{pyne_pyne_2011} are fertile ground 
for contributing computational nuclear science and reactor physics software 
tools with a wide impact potential. My research program will leverage the 
significant effort represented by those tools, rather than reinventing the 
wheel, and will benefit from the user and developer communities that they 
possess.  I will briefly expand on my near term focus areas which 
exemplify these goals. 

%My past and current work in the area of nuclear fuel cycle analysis has focused 
%on the application and development of the Cyclus nuclear fuel cycle simulator 
%project \cite{cyclus_github_2011} and of the Cyder used fuel disposition and 
%disposal system model \cite{cyder_github_2012, huff_integrated_2013}. Most 
%recently, I have extended my research focus to include neutronics and coupled 
%physics modeling of the Pebble-Bed, Fluoride-Salt-Cooled, High-Temperature 
%Reactor (PB-FHR) \cite{facilitators_fluoride-salt-cooled_2013, 
%facilitators_fluoride-salt-cooled_2013-1, 
%facilitators_fluoride-salt-cooled_2013-2, 
%facilitators_fluoride-salt-cooled_2013-3}.  My previous research has included 
%numerical methodologies for accelerator physics applications 
%\cite{huff_single_2003, huff_digital_2004}, experimental cosmological telescope 
%calibration \cite{huff_celestial_2008}, and experimental condensed matter 
%physics \cite{clerc_liquid_2008}. 
%
%What got you interested in this research?
%What was the burning question that you set out to answer?
%How can your research be applied?
%Why is your research important within your field?
%What challenges did you encounter along the way, and how did you overcome them

%\paragraph{Experimental Physics Research}
% Los Alamos
% QUIET
% Universidad de Chile


\paragraph{Coupled Multiphysics for Advanced Reactor Analysis}

The development of a high-fidelity, fully-coupled, multiphysics modeling toolkit 
for accident transient scenarios will be the main focus of my research. In 
particular, simulations capturing Design- and Beyond-Design-Basis Accident 
transients are essential to the advancement of nuclear reactor safety. Faithful 
assessments of reactor response in these scenarios require fully coupled, 
transient simulation of neutronics, thermal hydraulics, and structural 
performance, necessitating specialized computational methodologies.  


In particular, the Jacobian-Free Newton-Krylov (JFNK) solution method 
\cite{knoll_jacobian-free_2004}, combined with 
physics-based preconditioning, enables extraordinary parallel scalability
and fully-coupled solutions to systems of neutron transport and thermal 
hydraulic equations. The Multiphysics 
Object Oriented Simulation Environment (MOOSE) \cite{gaston_moose:_2009} MOOSE, 
from Idaho National Laboratory (INL) relies on this exceptional numerical 
method as well as adaptive mesh refinement for structured and 
unstructured meshes. This tool is beginning to be used in earnest for reactor physics 
applications (e.g., \cite{park_tightly_2009, short_multiphysics_2013, 
novascone_assessment_2012, novascone_multidimensional_2012, 
gaston_parallel_2009} among others). Moreover, the MOOSE tool possesses a 
modular, object-oriented simulation environment approach, similar to that of 
Cyclus, which allows user-developers to construct applications by focusing on 
the unique physics of their modeling problem  with minimal concern for the 
system solution methodology. 

I am especially interested in the potential impacts of the fully coupled 
nature of MOOSE multiphysics on accuracy in accident transient scenarios, which 
can be sensitive to coupling distortions. The ability to eliminate or nearly 
eliminate the scaling and coupling distortions seen in simulations that are only 
loosely or tightly coupled is a breakthrough capability that could change the 
face of reactor modeling entirely.  

In collaboration with BIDS, my future work with MOOSE, will continue a recent 
collaboration that I forged with the MOOSE team. This work focuses on the 
development of MOOSE software `Applications' enabling 3-dimensional, 
multi-scale, massively parallel analyses of promising reactor technologies. This 
will consist of developing dimension-agnostic, application-driven physics 
`Kernels,' combining them with validated kernels developed by others, and 
constructing them into coherent simulation objects. 

I will emphasize generalized methods for representing the neutronics of more 
arbitrary reactor designs, but with particular focus on those reactor 
technologies boasting inherent safety features (i.e. accident tolerant fuels or 
non-voiding coolants).  The first reactor design to be modeled is the PB-FHR, a 
Pebble-Bed, Fluoride-salt-cooled, High-temperature Reactor.  By relying on work 
complete so far with the PRONGHORN module within the MOOSE ecosystem, this work 
will result in a transient, three-dimensional MOOSE module with fully coupled 
thermal-hydraulics and deterministic neutronics.  

\paragraph{PyNE Development}
Throughout this work, an open source framework for nuclear science and
engineering, PyNE, will will be extended with assistive utilities necessary to
accomplish the reactor concept detail in multiphysics simulations above. 

For example, the complex meshes representing reactor geometries will be
generated using the PyNE mesh utility. As this work is conducted, PyNE will be
improved to more generally represent non-lattice geometries, such as pebble bed
and other fluidized fuel geometries. 

As the multiphysics applications described above are added to the MOOSE
ecosystem, a suite of common needs and components will emerge.  In support of
the broader needs of the open source nuclear science and engineering community,
these tools will be contributed to PyNE where applicable.  Providing these
capabilities in this fashion has two substantial benefits to the PyNE
framework.  In addition to reducing overall development effort by reuse of
robust solutions already existing in PyNE, it allows dependent tools (input,
output, and visualization, for example) to depend on a common infrastructure.

For instance, while a robust Material class within the PyNE toolkit
currently provides a specification for arbitrary isotopic vectors, there is no
class responsible for specifying the geometric placement of those materials in
a reactor core. However, to capture these and other details important to the
differentiation of reactor concepts, libraries will be provided to enable
specification of device architectures (i.e. reactor core configurations),
neutronics parameters (e.g., target burnups, multiplication factors), and other
details.


\section*{\textcolor{gray}{\it Impact Potential}}
Computational nuclear engineering relies on enormous datasets in many 
dimensions, traverses disparate scales, incorporates many physics, and demands 
precision.  Nuclear data typically takes the form of large libraries involving 
tabulated and evaluated data representing neutron, charged particle, and atomic 
reaction probabilities for all of the more than three thousand known 
radionuclides.  The driving equation for neutron behavior, the time-dependent 
Boltzmann equation, is solved in a 7-dimensional phase space, ($3$ in space, $2$ 
in angle, and one each in energy and time). The scale of a nuclear reactor 
simulation is inherently large, spanning five orders of magnitude in space and 
ten in neutron energy. Resolved discretization across those scales would require 
over $10^{17}$ degrees of freedom per timestep, well beyond the
capabilities of even exascale computing. In this way, without sophisticated
data and analysis methods, we would run out of computational resources before
heat transport, fluid flow, or material performance in the reactor had even
been addressed. 

I expect that applying modern data methodologies will lead to breakthrough 
advancements in the accuracy of accident transient simulations in nuclear 
reactors. Additionally, community adoption of the PyNE nuclear data framework
could have an enormous impact on reproducibility in nuclear science and
engineering. Robust, validated contributions to this toolkit will certainly,
therefore, have a lasting effect on the field. 

\bibliographystyle{unsrt}
\bibliography{research}


\end{document}

