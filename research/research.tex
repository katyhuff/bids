\documentclass[a4paper, 10pt]{article}
\topmargin-1.5cm

\usepackage{fancyhdr}
\usepackage{pagecounting}
\usepackage[dvips]{color}

% Color Information from - http://www-h.eng.cam.ac.uk/help/tpl/textprocessing/latex_advanced/node13.html

% NEW COMMAND
% marginsize{left}{right}{top}{bottom}:
%\marginsize{3cm}{2cm}{1cm}{1cm}
%\marginsize{0.85in}{0.85in}{0.625in}{0.625in}

\advance\oddsidemargin-0.85in
\advance\evensidemargin-0.85in
\textheight9.1in
\textwidth6.75in
\newcommand\bb[1]{\mbox{\em #1}}
\def\baselinestretch{1.25}
%\pagestyle{empty}
\newcommand{\hsp}{\hspace*{\parindent}}
\definecolor{gray}{rgb}{0.4,0.4,0.4}

\newcommand{\authorname}{Kathryn~D.~Huff }
\newcommand{\authoremail}{katyhuff@gmail.com}
\newcommand{\authorsite}{katyhuff.github.com}

\begin{document}
\pagestyle{fancy}
%\pagenumbering{gobble}
%\fancyhead[location]{text} 
% Leave Left and Right Header empty.
%\lhead{}
%\rhead{}
%\rhead{\thepage}
\lhead{\textcolor{gray}{\it \authorname}}
\rhead{\textcolor{gray}{\thepage/\totalpages{}}}
\renewcommand{\headrulewidth}{0pt} 
\renewcommand{\footrulewidth}{0pt} 
\fancyfoot[C]{\footnotesize \textcolor{gray}{\authorsite}} 

\begin{center}
{\LARGE \bf Research Project}\\
\vspace*{0.1cm}
{\normalsize \authorname (\authoremail)}
\end{center}
%\vspace*{0.2cm}

%\begin{document}
%\centerline {\Large \bf Research Statement for \authorname}
%\vspace{0.5cm}

% Write about research interests...
%\footnotemark
%\footnotetext{Check This}


% Introduction
Improving the safety and sustainability of nuclear power requires improved 
nuclear reactor designs, fuel cycle strategies, and waste disposal concepts.  
These systems are sufficiently complex that modern software methods and 
high-performance computing resources are essential to understanding and 
improving them. The importance of sophisticated scientific computing in 
advancing safe, sustainable nuclear power drives my research direction toward 
the development of computational methods and simulation tools for nuclear 
science and  reactor physics across many scales, with special focus on the 
safety and security of advanced reactor designs and fuel cycles.

Computational nuclear engineering relies on enormous datasets in many 
dimensions, traverses disparate scales, incorporates many physics, and demands 
precision.  Nuclear data typically takes the form of large libraries involving 
tabulated and evaluated data representing neutron, charged particle, and atomic 
reaction probabilities for all of the more than three thousand known 
radionuclides.  The driving equation for neutron behavior, the time-dependent 
Boltzmann equation, is solved in a 7-dimensional phase space, ($3$ in space, $2$ 
in angle, and one each in energy and time). The scale of a nuclear reactor 
simulation is inherently large, spanning five orders of magnitude in space and 
ten in neutron energy. Resolved discretization across those scales would require 
over $10^{17}$ degrees of freedom per timestep, well beyond the
capabilities of even exascale computing. In this way, without sophisticated
data and analysis methods, we would run out of computational resources before
heat transport, fluid flow, or material performance in the reactor had even
been addressed. 

%correct a dearth of accuracy, transparency, and access in the current landscape of computational tools for nuclear engineering. 
In the near term, I intend to continue developing computational methods for fuel 
cycle sustainability and reactor safety evaluation. These are the areas in which I 
have had sufficient experience to independently develop and pursue relevant 
research questions.  Can nonlinear programming techniques be applied to 
determine appropriately critical fresh fuel compositions in a reprocessing 
facility model? To what extent are repository site selection decisions and fuel 
cycle decisions coupled? What are the comparative global fuel supply impacts of 
political instabilities, international fuel takeback agreements, or
trade embargoes between specific pairs of nations?  What deterministic methods 
best capture the physics of complex geometries such as pebble fuel with TRISO 
particles? What is the effect of this fuel choice on the bounding magnitude of a 
prompt reactivity insertion resulting in safe shutdown? 

In the longer term, I hope to establish a research program capable of tackling a 
variety of challenges in nuclear energy from the technical to the political.  I 
hope to uncover groundbreaking insights by means of simulation methods, 
efficient algorithms, novel computer architectures, and sophisticated design 
tailored to the many scales and multiple physics encountered in nuclear energy 
technology.  The tenets of scientific endeavor (e.g., data control, 
reproducibility, comprehensive documentation, and peer review) suffer in 
projects that fail to make use of current development tools such as unit 
testing, version control, automated documentation and others 
\cite{wilson_best_2014, merali_computational_2010}. Maintenance of rigorous 
scientific computing standards will help my research program avoid such pitfalls 
\cite{huff_rapid_2011} and incorporation of new numerical methods for reactor 
physics will keep my research program relevant.

\section*{\textcolor{gray}{\it Past and Current Research}}

My past and current work in the area of nuclear fuel cycle analysis has focused 
on the application and development of the Cyclus nuclear fuel cycle simulator 
project \cite{cyclus_github_2011} and of the Cyder used fuel disposition and 
disposal system model \cite{cyder_github_2012, huff_integrated_2013}. Most 
recently, I have extended my research focus to include neutronics and coupled 
physics modeling of the Pebble-Bed, Fluoride-Salt-Cooled, High-Temperature 
Reactor (PB-FHR) \cite{facilitators_fluoride-salt-cooled_2013, 
facilitators_fluoride-salt-cooled_2013-1, 
facilitators_fluoride-salt-cooled_2013-2, 
facilitators_fluoride-salt-cooled_2013-3}.  My previous research has included 
numerical methodologies for accelerator physics applications 
\cite{huff_single_2003, huff_digital_2004}, experimental cosmological telescope 
calibration \cite{huff_celestial_2008}, and experimental condensed matter 
physics \cite{clerc_liquid_2008}. 

%What got you interested in this research?
%What was the burning question that you set out to answer?
%How can your research be applied?
%Why is your research important within your field?
%What challenges did you encounter along the way, and how did you overcome them

%\paragraph{Experimental Physics Research}
% Los Alamos
% QUIET
% Universidad de Chile

\paragraph{Cyclus Fuel Cycle Simulator} Faced with inflexibility in previous 
nuclear fuel cycle system simulation frameworks, Paul P.H. Wilson and I 
envisioned a more modular, encapsulated design. I led this software design effort 
\cite{huff_open_2011, huff_cyclus_2011}, called Cyclus, and was its lead 
developer in its first years.  As a direct result of 
Cyclus' incorporation of modern software development practices and desired 
capabilities such as openness, modularity, and scalable fidelity, it has now 
grown into a multi-institution collaboration with dozens of users and 
contributors. 

\paragraph{Cyder Repository Analysis Model}

% what is cyder

My dissertation work \cite{huff_integrated_2013}, a hydrologic and thermal 
geologic disposal model called Cyder, illuminated the distinct dominant physics 
of candidate repository geologies, designs, and engineering components. By 
integrating hydrologic contaminant transport \cite{huff_sensitivity_2012} and 
transient thermal transport \cite{huff_numerical_2012} with the Cyclus fuel 
cycle simulation framework, Cyder calculates disposal-related fuel cycle metrics 
by providing a dynamic simulation environment in which to prototype geologic 
disposal concepts in the context of alternative fuel cycles 
\cite{huff_cyclus_2013}.  

% what is the methodology
%
%By implementing a number of interchangeable hydrologic contaminant transport 
%methods in a modular software design, this tool supports the examination of the 
%relative accuracy of those methods \cite{huff_cyclus_2013}. It also provides the 
%user with modeling options that alternately optimize speed and sensitivity for 
%an array of reducing, saturated geologic disposal environments (i.e. enclosed 
%clay, salt, and granite concepts).  Sensitivity analyses of a detailed 
%hydrologic contaminant transport repository model were conducted to inform of 
%abstraction efforts necessary for the development of these semi-analytical 
%models \cite{huff_sensitivity_2012}.  
%
%A transient thermal model capable of arriving at a heat-based capacity quickly 
%for an arbitrary waste stream enables the model to perform dynamic thermal 
%capacity estimation.  It relies on tabulated repository heat evolution results 
%for a parameter space covering the thermal coefficient range of the main 
%geologies of interest and over a range of realistic waste package spacings. The 
%analytic model used to calculate the tabulated results was developed at Lawrence 
%Livermore National Laboratory and compared favorably to traditional finite 
%difference models during a benchmarking effort that I conducted jointly with 
%Theodore H. Bauer at ANL \cite{huff_numerical_2012}.

%\paragraph{Other Fuel Cycle Analysis Activities}
%
%With the Systems Analysis Campaign at Idaho National Laboratory I applied the 
%vision for Cyclus' ongoing development and lessons learned from previous 
%simulators to  author a requirements document intended to direct development for 
%their Grand Challenge Fuel Cycle Simulator initiative \cite{huff_next_2010}.   
%Additionally, in a Non-Proliferation Case Study within the University of 
%Wisconsin committee on World Affairs and Global Energy (WAGE), I corresponded 
%with experts and researched modeling methodologies in use in quantitative 
%political science as well as qualitative analysis and experimental models.  This 
%case study investigated the applicability of analytical models for determining 
%indicators that a nation-state is likely to begin a nuclear weapons program. 

\paragraph{Coupled Multiphysics for Advanced Reactor Analysis}

My postdoctoral research has begun to expand my expertise by emphasizing the 
development of a high-fidelity, fully-coupled, multiphysics model of the PB-FHR 
at the University of California - Berkeley with Professors Jasmina Vuji\`c and Per 
Peterson. The objective of this work is to conduct high-fidelity simulations of 
transient event behavior in this complex reactor in a high-performance computing 
environment.  In my first months at Berkeley, I forged a collaboration with the 
Idaho National Laboratory to conduct this development within the Multiphysics 
Object Oriented Simulation Environment (MOOSE) \cite{gaston_moose:_2009}. This 
work will result in a transient, three-dimensional MOOSE module with fully 
coupled thermal-hydraulics and deterministic neutronics.  


%\paragraph{Physics Collaborations} 
%In collaboration with the Los Alamos Neutron Science Center, I conducted two 
%projects; in one, I employed digital filtration techniques to the neutron cross 
%section data from the Lead Slowing Down Spectrometer and, in the other, I built 
%MCNPX models of an experimental apparatus for detection of neutron induced 
%electronics failures . This was followed by artificial and celestial calibration 
%of polarimeters for the Q and U Imaging ExperimenT (QUIET) cosmic microwave 
%background telescope at the University of Chicago \cite{huff_celestial_2008} . 
%Finally, I collaborated on a far from equilibrium granular materials experiment 
%in which an experimental apparatus that I built was used to compare the van der 
%Waals gas model to phase changes in a spatially constrained collection of 
%rapidly shaken metal spheres \cite{clerc_liquid_2008}. 

\section*{\textcolor{gray}{\it Future Research}}
%What direction will your research take you in next
%what new questions do you have?
% external research funding
Novel solutions to challenges in nuclear energy such as nuclear reactor accident 
response, nuclear fuel cycle strategy, and waste management will arise from the  
development of application-driven numerical simulation methods and incorporation 
of sophisticated computational tools. To this end, my work will range from 
individual contributions to existing toolkits to comprehensive solution 
libraries for specific nuclear applications (e.g., a suite of activation 
analysis methods for sub-critical systems). Open source projects such as 
Cyclus \cite{huff_cyclus_2011}, MOOSE \cite{gaston_moose:_2009}, and the Python 
for Nuclear Engineering (PyNE) toolkit \cite{pyne_pyne_2011} are fertile ground 
for contributing computational nuclear science and reactor physics software 
tools with a wide impact potential. My research program will leverage the 
significant effort represented by those tools, rather than reinventing the 
wheel, and will benefit from the user and developer communities that they 
possess.  I will briefly expand on two of my near term focus areas which 
exemplify these goals. 

\paragraph{Advanced Technology Modeling for Fuel Cycle Analysis} 

One of the continuing challenges in evaluating the system level impacts of 
nuclear technologies and policies is the need for a standardized simulation 
platform and library of models representing nuclear technologies and calculating 
fuel cycle metrics.  The Cyclus simulation framework 
has provided this standardizing platform, but it is incomplete without a 
robust library of fuel cycle metric calculation methods and a versatile ecosystem of 
models to represent innovative nuclear technologies. My work in this area will 
fill this gap. 

The Cyclus modeling paradigm enables both scientific and policy analyses by 
following transactions of discrete quanta of material among discrete facilities, 
arranged in a geographic and institutional framework, and trading in flexible 
markets. Cyclus' sophisticated design emphasizes separation of the core simulation logic 
from the technical nuclear process models representing the facilities of the 
nuclear fuel cycle. This ensures robust modularity with regard to functionality.  

My work will leverage Cyclus toward the development of a library 
of methods for calculating fuel cycle metrics.  Object-oriented methods enabled 
by the unique Cyclus modeling paradigm will result in metrics with richer detail 
than has historically been possible with previous simulators. One example is a 
``shadow fuel cycle'' nonproliferation calculation strategy in which nuclear material is 
diverted by a nefarious actor within a facility to determine the minimum 
detectable diverted material amount.  This metric is used to describe the 
detection sensitivity of inspections.  Creative safeguards strategies may emerge 
from the richness of disaggregated detail made possible by evaluating this 
nefarious theft scenario within the context of the discrete materials and 
discrete facilities of the Cyclus environment.

Additionally, my work will include both focused, single technology model development and 
development of concise parametric models capable of representing myriad 
technologies.  Such models will enable assessment of a broad range of technology 
and policy implications related to the introduction of those technologies into 
the world nuclear energy market. By balancing speed with the capability to 
capture dominant physics, Cyclus can rapidly assess the implications of those 
technologies on the broader fuel cycle.  Contributing advanced reactor and fuel 
cycle concepts to the ecosystem of available models will also increase the 
potential for collaboration with other domestic and international researchers in 
technology evaluations and community benchmarking exercises.

I envision, for example, advising a student interested in innovative reactor 
design to develop a parameterized spectral model to approximate burnup and 
transmutation physics with appropriate speed and fidelity for fuel cycle 
simulations.  This work will benefit from my experience navigating this 
trade-off between speed and fidelity, in which a model must be rapid while 
simultaneously detailed enough to uncover system level responses to technology 
choices.


\paragraph{Multiphysics Model Development for Reactor Design and Analysis}

Another focus of my research program will include the development of 
multiphysics models of reactor designs, with particular focus on those boasting 
inherent safety features (i.e. accident tolerant fuels or non-voiding coolants).  
Design- and Beyond-Design-Basis Accident simulations are essential to the 
advancement of nuclear reactor safety. Faithful assessments of reactor response 
in these scenarios require fully coupled, transient simulation of neutronics, 
thermal hydraulics, and structural performance, necessitating specialized 
computational methodologies. 

In particular, the Jacobian-Free Newton-Krylov (JFNK) solution method 
\cite{knoll_jacobian-free_2004}, combined with 
physics-based preconditioning, enables extraordinary parallel scalability and 
and fully-coupled solutions to systems of neutron transport and thermal 
hydraulic equations. MOOSE, from INL, which relies on this exceptional numerical 
method as well as adaptive mesh refinement for structured and 
unstructured meshes, is beginning to be used in earnest for reactor physics 
applications (e.g., \cite{park_tightly_2009, short_multiphysics_2013, 
novascone_assessment_2012, novascone_multidimensional_2012, 
gaston_parallel_2009} among others). Moreover, the MOOSE tool possesses a 
modular, object-oriented simulation environment approach, similar to that of 
Cyclus, which allows user-developers to construct applications by focusing on 
the unique physics of their modeling problem  with minimal concern for the 
system solution methodology. 

My future work with MOOSE, continuing a recent collaboration that I forged with the 
MOOSE team while at UC-Berkeley, will focus on the development 
of MOOSE `Applications' capable of 3-dimensional, multi-scale, massively parallel 
analyses of promising reactor technologies. This will consist of developing 
dimension-agnostic, application-driven physics 
`Kernels,' combining them with validated kernels developed by others, and 
constructing them into coherent simulation objects. 

In particular, I am interested in the potential impacts of the fully coupled 
nature of MOOSE multiphysics. The ability to eliminate or nearly eliminate the 
scaling and coupling distortions seen in simulations that are only loosely or 
tightly coupled is a breakthrough capability that could change the face of 
reactor modeling entirely.  

\paragraph{External Research Funding}
My future research plans fit well into the Department of Energy (DOE) Nuclear 
Energy University Programs (NEUP) funding scope. In particular, both Cyclus and 
MOOSE applications have been supported in recent workscopes. Since Cyclus' 
recent adoption by the DOE Office of Nuclear Energy (DOE-NE)  as its flagship 
nuclear fuel cycle simulator, it has garnered an international, 
multi-institution collaboration of users and developers including more than 
three NEUP-funded projects. 

The inclusion of the Cyclus project in previous workscopes indicates that 
nuclear fuel energy system simulation development with the Cyclus tool is an 
area of sustained interest within the DOE-NE.  I have already submitted, as 
principal investigator, a 2014 NEUP pre-proposal to support work on advanced 
laser separation technology modeling for fuel cycle analysis.  I expect that my 
intimate familiarity with Cyclus will add strength to this and future proposals.  

Nonproliferation applications of fuel cycle analysis have the potential to benefit greatly from funding 
opportunities through the National Nuclear Security Administration (NNSA), 
Nuclear Regulatory Commission (NRC), and Department of Homeland Security 
Domestic Nuclear Detection Office (DHS-DNDO). These pursuits in combination with 
my focus on safety may  strengthen my applications to junior faculty development 
awards available from the National Science Foundation (NSF), DOE, DHS-DNDO, the 
Office of Naval Research Research (ONR), and the Air Force Office of Scientific 
Research (AF-OSR).

Finally, I also expect to seek additional support for methods development and 
contribution to computational nuclear toolkits from programs such as Google 
Summer of Code, the Alfred P. Sloan Foundation, and the Gordon and Betty Moore 
Foundation.  The NSF is also interested in 
scientific computing and I am currently assisting in the collaborative preparation 
of a proposal for funding from NSF in the field of scientific computing 
education. Furthermore, collaboration with Nuclear Engineering Advanced 
Modeling and Simulation campaign with current and past colleagues at Idaho, 
Argonne, and Oak Ridge National Laboratories may provide additional student and 
collaboration support in the area of modeling and simulation. 

I have had many opportunities to participate in both failed and successful grant 
proposals.  As a graduate student, I assisted in writing grant proposals with my 
advisor, Paul P.H. Wilson. In 2012, one of those reached success, securing a 
three year, \$1.2 million, Nuclear Energy University Programs (NEUP) Research 
and Development grant for the expansion of Cyclus. Additionally, I was the 
primary author on an NEUP proposal for the 2013 call for proposals. I was 
pleased to have been invited to submit a full proposal, but that full proposal 
did not progress further. It is my hope that these early experiences with grant 
writing will serve me well in my own research program.




\bibliographystyle{unsrt}
\bibliography{research}

\end{document}

