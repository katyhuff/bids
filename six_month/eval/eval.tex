\documentclass[a4paper, 10pt]{article}
%\topmargin-1.5cm

\usepackage{fancyhdr}
\usepackage{pagecounting}
\usepackage[dvips]{color}

% Color Information from - http://www-h.eng.cam.ac.uk/help/tpl/textprocessing/latex_advanced/node13.html

% NEW COMMAND
% marginsize{left}{right}{top}{bottom}:
%\marginsize{3cm}{2cm}{1cm}{1cm}
%\marginsize{0.85in}{0.85in}{0.625in}{0.625in}

%\advance\oddsidemargin-0.85in
%\advance\evensidemargin-0.85in
%\textheight8.5in
%\textwidth6.75in
\newcommand\bb[1]{\mbox{\em #1}}
\def\baselinestretch{1.25}
%\pagestyle{empty}
\newcommand{\hsp}{\hspace*{\parindent}}
\definecolor{gray}{rgb}{0.4,0.4,0.4}

\newcommand{\authorname}{Katy Huff}
\newcommand{\longauthorname}{Dr.  Kathryn~Huff}
\newcommand{\authorsite}{katyhuff.github.io}


\newcommand{\myitem}[1]{\item[\textcolor{gray}{\textbf{#1}}]}
\newcommand{\boldblue}[1]{\textcolor{cyan}{\textbf{#1}}}

\begin{document}
\pagestyle{fancy}
%\pagenumbering{gobble}
%\fancyhead[location]{text}
% Leave Left and Right Header empty.
%\lhead{}
%\rhead{}
%\rhead{\thepage}
\lhead{\textcolor{gray}{\it \authorname}}
\rhead{\textcolor{gray}{\thepage/\totalpages{}}}
\renewcommand{\headrulewidth}{0pt}
\renewcommand{\footrulewidth}{0pt}
\fancyfoot[C]{\footnotesize \textcolor{gray}{\authorsite}}

\begin{center}
{\LARGE \bf Self Evaluation}\\
\vspace*{0.1cm}
{\normalsize \longauthorname}
\end{center}
%\vspace*{0.2cm}


% Introduction

The following is a non-exhaustive list of activities I have participated in 
since arriving in Berkeley in September 2013. During that time, I have been, on 
average funded by the FHR project ($\sim$30\%), NSSC 
($\sim$30\%), BIDS ($\sim$25\%), and 
LLNL ($\sim$15\%). \boldblue{Especially notable accomplishments are in blue}.

\section*{FHR-Related}

\begin{itemize}
\myitem{PyRK} I have written a \boldblue{Python package for 0-D accident transient 
modeling in nuclear reactors (PyRK) \cite{huff_pyrk_2015}}. I have conducted an 
SFR analysis for validation. Additionally, this tool is expected to be 
used as an engine for running accident transient experiments in CIET this fall. 
\myitem{PB-FHR Analysis} The main purpose of PyRK, however, is simulation of 
Pebble-Bed Fluoride-Salt-Cooled High-Temperature Reactor (PB-FHR).  transients.  
Accordingly, I am working on a manuscript related to my current results from 
PyRK for the case of reactivity insertions and Loss of Heat Sink (LOHS) 
transients in the PB-FHR. I will soon distribute a draft to my collaborators 
for review and participation in the hopes that it will ideally be submitted 
late this summer.
\myitem{MOOSE Extension} Development of my 3D, multi-scale, multi-physics PBFHR 
model has begun. Using Pronghorn and RattleSNake within the MOOSE framework, I 
can couple thermal hydraulics on coupled coarse and fine meshes. This requires 
me to modify the pebble-bed flow-model to (currently gaseous flow) to allow 
molten salt coolant in the pebble bed. I expect this analysis software will be 
\boldblue{ready for demonstration on the  NERSC resources in late fall 2015.} 
\myitem{NERSC} I am pleased that, based on a proposal I submitted with BIDS, 
\boldblue{I was awarded a signficant time allocation on NERSC.} I intend to 
use my NERSC allocation (millions of cpu hours) to conduct my PB-FHR transient 
analysis this fall (MOOSE extension above).
\myitem{INL LDRD} Related to my MOOSE extension, I am a co-investigator on an 
LDRD proposal that, if funded, will provide travel funding to INL, access to 
potenially validating data from the industry co-investigator, and potential 
summer funding for a Berkeley student.
\myitem{COMSOL ATWS} I refactored Scarlat's AHTR COMSOL model to include the geometry 
and neutronics of the PB-FHR and ran the first PB-FHR ATWS analysis. The 
pressure drop optimization was conducted by Huddar and our results were 
included in the PB-FHR Mk1 Design Report.  
\myitem{BDBE Workshop} I led analysis for the Source Term Analysis and 
Radiological Release Pathways section of the never-released 2013 BDBE workshop 
white paper. I also assisted with Beyond Design Basis Event Analysis 
Methods and Experimental Gaps section of the white paper.
\end{itemize}

\section*{Other Nuclear Engineering}
\begin{itemize}
\myitem{PyNE} I am a contributor to the PyNE (python for nuclear engineering 
toolkit). Accordingly, I co-authored a PyNE ANS conference paper 
\cite{bates_pyne_2014} and hosted two PyNE hackathons (NSSC 2013 and BIDS 
2014). In the hackathons, we had developers join us for a few days and we 
improved this open source package immensely. \boldblue{This package is used by many 
nuclear engineers at universities and the national laboratories (including some 
in the FHR group at Berkeley).}
\myitem{Cyclus} Continuing my involvement, I contributed to the most recent 
release od Cyclus and have helped to conduct Fuel Cycle analyses in 
collaboration with DOE and Prof. Fratoni. Finally, I \boldblue{submitted two 
manuscripts} this year related to my past work with the Cyclus project. Though 
they have not yet been accepted, they are in revision.
\myitem{Cyder} I \boldblue{submitted and resubmitted a journal article} based 
on my dissertation work concerning hydrologic and thermal modeling of nuclear 
waste repositories. 
I am awaiting review comments on the resubmitted paper.
\myitem{FCWMD Vice-Chair} Continuing my service to ANS, I am now the 
\boldblue{Vice-Chair of the Fuel Cycle and Waste Management Division.}
\myitem{Conference Papers} I submitted an ANS summary on the topic of 
a nuclear engineering course syllabus based on my recently published book.
\myitem{BFF Program} I was invited to and attended a ``Building Future Faculty
Program'' at NCSU. This has already contributed to my pursuit of a faculty 
position in Nuclear Engineering.
\myitem{Mentorship} To varying degrees, I have guided the research computation 
of numerous students in the NE department including Xin Wang, Blake Huff, Tommy 
Cisneros, Ryan Bergmann, Kelly Rowland, Madicken Munk, Grant Buster, Josh 
Howland, and Russell Nibbelink. 
\end{itemize}


\section*{Scientific Computing Education}

\begin{itemize}
\myitem{Best Practices} In 2014, I coauthored an extremely popular paper on how 
best to use computers in science\cite{wilson_best_2014}. It has been 
\textcolor{cyan}{\textbf{cited 80 times}}.
\myitem{The Hacker Within} I have led a \boldblue{popular weekly seminar on 
scientific computing, attended by many nuclear engineering graduate students} 
I've brought dozens of people into the NSSC and BIDS spaces with this meeting. 
For this seminar, I schedule tutorials on tools and best practices for 
scientific computing. It is popular among nuclear engineers and physicists, but 
attracts a diversity of individuals. This recent success has inspired my PhD 
advisor to reboot the original THW in Madison and Dr. Arna Karnik at Swinburne 
University in Melbourne, Australia has started her own chapter of THW there 
within the physics department.
\myitem{Software Carpentry} I am the current \boldblue{elected Chair 
of the Software Carpentry Foundation Steering Committee}. This international 
nonprofit organization focuses on teaching scientific computing skills to 
scientists.  Software Carpentry is responsible for over 100 workshops per year 
and now has dozens of university, laboratory, and governmental partners.  My 
leadership has additionally led to a BIDS collaboration with them and has 
facilitated sold out workshops in the BIDS space. 
\myitem{Case Studies} With the BIDS Reproducibility Working Group, I
have helped collect case studies of reproducible workflows in scientific work 
on campus. As a result, I am on track to be a \boldblue{chapter author} on the book form 
of this collection and look forward to co-authoring an extended whitepaper or 
journal article on the lessons learned.
\myitem{MOOSE Workshop} Professor Fratoni and I have arranged for a workshop on a
multiphysics simulation environment, MOOSE.
\myitem{WiSE Workshop} I was the lead instructor for a workshop at LBNL 
dedicated to women in science and engineering. It became a 
\myitem{GitHub Town Hall} I was invited by GitHub to visit the UW eScience space
and to sit on a panel discussing ``What Academia Can Learn From Open Source''.
\myitem{O'Reilly Book} Between May 2014 and January 2015, \boldblue{I wrote a 
book} to help students and researchers in the physical sciences to conduct the 
computational aspects of their research more effectively. It's called Effective 
Computation In Physics and hundreds of copies have already been sold.
\myitem{Guest Lectures} I have served as a guest lecturer for many lessons in 
NE155 and NE255.
\end{itemize}


\section*{Other}
\begin{itemize}
\myitem{SciPy} I have served as the Technical Program Chair (2013 and 2014) and 
Proceedings Chair (2015) for SciPy, a conference on the scientific use of 
python that brings together an entire community at the intersection of science 
and programming.
\myitem{ASPP School} I have been invited two years in a row to be the 
\boldblue{keynote speaker} for a week-long Advanced Scientific Programming in 
Python Summer School in Europe (Croatia 2014, Munich 2015).
\myitem{BSN Deep Dive} I was invited to be a mentor for a weekend 
retreat intended to help female Berkeley graduate students in STEM.  This is 
called a ``deep-dive'' and was hosted by the Berkeley Science Network at 
Asilomar in March 2015.
\myitem{NEUP 2013 Proposal} I was primary author, but not PI on an NEUP 
Proposal on reactor technology analysis in the context of fuel cycles that was 
invited back for a full proposal but ultimately was not awarded.  
\myitem{NEUP 2014 Proposal} I was PI on an NEUP Proposal on laser isotopic 
separation that was invited back for a full proposal but ultimately was not 
awarded.
\myitem{NSF 2014 Proposal} I was a co-investigator on an NSF proposal (led by 
Professor Slaybaugh) related to scientific computing education. It was ultimately not 
awarded.
\end{itemize}
\bibliographystyle{plain}
\bibliography{eval}
\end{document}
