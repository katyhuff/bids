\documentclass[a4paper, 10pt]{article}
%\topmargin-1.5cm

\usepackage{fancyhdr}
\usepackage{pagecounting}
\usepackage[dvips]{color}

% Color Information from - http://www-h.eng.cam.ac.uk/help/tpl/textprocessing/latex_advanced/node13.html

% NEW COMMAND
% marginsize{left}{right}{top}{bottom}:
%\marginsize{3cm}{2cm}{1cm}{1cm}
%\marginsize{0.85in}{0.85in}{0.625in}{0.625in}

%\advance\oddsidemargin-0.85in
%\advance\evensidemargin-0.85in
%\textheight8.5in
%\textwidth6.75in
\newcommand\bb[1]{\mbox{\em #1}}
\def\baselinestretch{1.25}
%\pagestyle{empty}
\newcommand{\hsp}{\hspace*{\parindent}}
\definecolor{gray}{rgb}{0.4,0.4,0.4}

\newcommand{\authorname}{BIDS Fellow}
\newcommand{\longauthorname}{BIDS Fellow}
\newcommand{\authorsite}{bids.berkeley.edu}

\begin{document}
\pagestyle{fancy}
%\pagenumbering{gobble}
%\fancyhead[location]{text}
% Leave Left and Right Header empty.
%\lhead{}
%\rhead{}
%\rhead{\thepage}
\lhead{\textcolor{gray}{\it \authorname}}
\rhead{\textcolor{gray}{\thepage/\totalpages{}}}
\renewcommand{\headrulewidth}{0pt}
\renewcommand{\footrulewidth}{0pt}
\fancyfoot[C]{\footnotesize \textcolor{gray}{\authorsite}}

\begin{center}
{\LARGE \bf BIDS Feedback}\\
\vspace*{0.1cm}
{\normalsize \longauthorname}
\end{center}
%\vspace*{0.2cm}


% Introduction

\section{\textcolor{gray}{What are the greatest benefits of being a BIDS data science fellow, and what are the greatest challenges?}}

The benefits are numerous.
\begin{itemize}
\item The people (fellows, of course, but also Kevin, Ali, Saul, and Fernando) are wonderful
\item The space is wonderful
\item The ability to host meetings in the space is priceless (really, this is the biggest thing for me)
\item I’ve learned some data science vocabulary
\item BIDS gives me a strong backup plan (data science in the bay) if I don't get a faculty position
\item The coffee machine is great
\item I’ve had an excuse to do things I think are wonderful and important but don't fit into my science domain (this is also a downside, see: “distraction”)
\end{itemize}

The greatest challenge is mostly the expectation that, in addition to our own
research goals and career goals, the fellows are expected to do an insane
amount to lead this institution without much support at all from the senior
PIs. The general expectations are high and I feel a lot of pressure to:

\begin{itemize}
\item Be involved in tons of working groups (I tried to say ``no I'm only going
to do two'', but other fellows and even senior PIs encourage me to join theirs
too\ldots)
\item Lead working groups (mostly without any guidance whatsoever)
\item Initiate collaborations in the space
\item Attend 2-4 meetings and events per week
\item Identify, invite, and communicate with all the tea speakers
\item Advertise teas, lectures, etc. so that the space isn't empty (que horror)
\item Set up and break down the space for activities like teas and THW (can we please get the undergrad work-study person to do some of this?)
\item Attend retreats, methods talks, meetings with industry folks, faires, etc.
\end{itemize}

It's too much for 14 fellows at 50\% time to succeed at. I'm hoping that when
there are more fellows, it will be a little easier to split up some of these
responsibilities.



\section{\textcolor{gray}{Are there concrete ways that BIDS could better
support your needs as a fellow or your career advancement?}}

My career will advance if just a couple of things appear:
\begin{itemize}
\item journal publications,
\item reference letters from experts in my domain,
\item successful grant proposals,
\item and appropriate faculty job openings.
\end{itemize}

\textbf{To help me succeed at producing journal publications,} BIDS could
support me by defending my precious time and by encouraging senior PIs to help
the working groups to emphasize publishable results.

\textbf{To help me succeed at getting reference letters from experts in my
domain,} BIDS could support me by communicating more closely with my PIs in my
home department. There was, at one point, a suggestion that we would have lunch
with our PIs and some senior PIs or something, to help to convince them I'm not
just goofing off. Perhaps BIDS could invite our PIs to activities where they
can interact with the senior fellows?

\textbf{To help with grants,} I would love BIDS to arrange a workshop
surrounding upcoming grant proposal opportunities that BIDS fellows and
affiliates might be able to apply for. In particular, I would love BIDS to call
upon the BIDS-Co-PIs to consider attending such workshops, leading discussions,
and considering leading some of these grant proposals. Since we are mostly
postdocs, none of us are going to succeed at being the PI on a big NSF
proposal. Even if we get exceptional PI status and don't flail helplessly in
the enormity of the process, the NSF is very unlikely to grant an award with a
postdoc as a PI. We need organized activities around these grant opportunities
and we need BIDS leadership to help us help them to win those grants. We cannot
do it on our own.

\textbf{Finally, as far as appropriate faculty job openings go,} I would apply
to a faculty position at Berkeley that is half Data Science and half My Domain.
It doesn't exist. The other DSE institutions have them, but not Berkeley. I'm
not interested in being a lecturer (not even with ``security of employment''), so
I’ll likely have to leave the Bay Area (and probably Data Science) forever in
Fall 2016.


\section{\textcolor{gray}{To what extent do you feel there has been a positive
synergy from having fellows work on separate projects at a common location as
opposed to, for example, all fellows working on a larger data science
problem?}}

Sure, it’s great. I love working in the space on my own thing. I’ve been able
to ask people questions about python packages and databases. Among BIDS
fellows, getting an answer to those questions is actually possible, which is
great.

\section{\textcolor{gray}{How well has the BIDS space worked? What are the good
and bad features of the space?}}

I love having the convertible central space for events and I love having the
little rooms for individual and group work. The open desks are great too, but
it would be great if those desks weren't so close to the central space (or if
they were separated somehow.)

I often feel like I have no choice but to participate in something happening in
the middle of the room. If I'm not interested in the friday lecture, it means I
have to leave BIDS and go to my other office, or try to go sit in one of the
little rooms like a hermit. For example, when Dani’s visitor from Dreamworks
came to visit, I really didn't feel like I was allowed to keep working at my
desk. People were actually dragged out of their meetings to attend this talk on
a random afternoon. It was a great talk, sure, but it’s not something I’ll ever
use and I lost almost two hours of important work time to it.

It would be great if there was just a general understanding that the fellows
should be allowed to just ignore tea or the friday lecture if they’re working
in the space. I currently worry that I’ll be scolded if I just continue to
work. I have never, for example, asked anyone in the space to join THW. Most
fellows just keep on working through that meeting.

\section{\textcolor{gray}{How effective have the BIDS events been?}}

\begin{itemize}
\item The short tea talks have been great. I really like them as long as they
stay short. I'm glad there is just one per week though.
\item I love THW (but I'm biased)
\item I have been blown away by some of the friday lectures
\item I didn't really love the spark workshop, but I was glad to have the opportunity to attend, and I learned some things
\item I really liked having John Canny's students in their poster session. I thought that was a lovely event.
\end{itemize}


\section{\textcolor{gray}{Any more general comments/feedback that you want to
capture here?}}

Some were mentioned above somewhat. I know I've expressed some negative things,
but I really am glad to be a BIDS fellow and I know everyone is trying really
really hard to help it succeed. Here is one idea to take away:

I'd love to see a 1-day workshop, led by a few interested BIDS co-PIs and
senior-fellows. If it's well organized, by the end of the day such a workshop
could result in:
BIDS teams interested in upcoming grant opportunities (with fellows as co-PIs!!)
and proposal outlines/drafts for some of those opportunities

\end{document}
