\documentclass[a4paper, 12pt]{article}
%\topmargin-1.5cm

\usepackage{fancyhdr}
\usepackage{pagecounting}
\usepackage[dvips]{color}

% Color Information from - http://www-h.eng.cam.ac.uk/help/tpl/textprocessing/latex_advanced/node13.html

% NEW COMMAND
% marginsize{left}{right}{top}{bottom}:
%\marginsize{3cm}{2cm}{1cm}{1cm}
%\marginsize{0.85in}{0.85in}{0.625in}{0.625in}

%\advance\oddsidemargin-0.85in
%\advance\evensidemargin-0.85in
%\textheight8.5in
%\textwidth6.75in
\newcommand\bb[1]{\mbox{\em #1}}
\def\baselinestretch{1.25}
%\pagestyle{empty}
\newcommand{\hsp}{\hspace*{\parindent}}
\definecolor{gray}{rgb}{0.4,0.4,0.4}

\newcommand{\authorname}{Reactor Design and Neutronics Group}
\newcommand{\longauthorname}{Dr. Kathryn~D.~Huff\\ Prof. Jasmina Vujic\\ Prof. 
Rachel Slaybaugh\\ and others.}
\newcommand{\authorsite}{ucb-rdn.github.com}

\begin{document}
\pagestyle{fancy}
%\pagenumbering{gobble}
%\fancyhead[location]{text} 
% Leave Left and Right Header empty.
%\lhead{}
%\rhead{}
%\rhead{\thepage}
\lhead{\textcolor{gray}{\it \authorname}}
\rhead{\textcolor{gray}{\thepage/\totalpages{}}}
\renewcommand{\headrulewidth}{0pt} 
\renewcommand{\footrulewidth}{0pt} 
\fancyfoot[C]{\footnotesize \textcolor{gray}{\authorsite}} 

\begin{center}
{\LARGE \bf Participation}\\
\vspace*{0.1cm}
{\normalsize \longauthorname}
\end{center}
%\vspace*{0.2cm}

%\begin{document}
%\centerline {\Large \bf Research Statement for \authorname}
%\vspace{0.5cm}

% Write about research interests...
%\footnotemark
%\footnotetext{Check This}

% Your response to the “call for participation” should be a 2-3 page document 
% that describes your efforts in data science, including your challenges and 
% opportunities. We are not asking for a budget or a formal proposal. Instead, 
% please provide an overview of the scientific goals and vision and how working 
% within a lager data science context will further that vision. Our intent is 
% two-fold:  To build a community of data science, and help us steer and evolve 
% the mission of BIDS; and to help attract funding from other sources, including 
% new grants, support from private donors, and corporate sponsorships.

% Introduction
The Reactor Design and Neutronics group at the University of California -- 
Berkeley \cite{reactor_ucb-rdn_2014} is a team of professors, postdocs, and 
students whose research is driven by the importance of sophisticated scientific 
computing in advancing safe, sustainable nuclear power.  
The direction of this group emphasizes the development of computational methods 
and simulation tools for nuclear reactor physics across many scales, with 
special focus on the safety and sustainability of advanced reactor designs and 
fuel cycles.  
It is our hope that novel solutions to these challenges in nuclear 
energy will arise from the development of application-driven numerical 
simulation methods and incorporation of sophisticated computational tools.  
We expect that involvement in the BIDS community has the potential to expand the 
scope of data and analysis methodologies within our wheelhouse. Similarly, we 
expect that our approach to data-driven science promises to similarly expand 
horizons for the rest of the BIDS community.

Computational nuclear engineering relies on enormous datasets in many 
dimensions, traverses disparate scales, incorporates many physics, and demands 
precision.  Nuclear data typically takes the form of large libraries involving 
tabulated and evaluated data representing neutron, charged particle, and atomic 
reaction probabilities for all of the more than three thousand known 
radionuclides.  The driving equation for neutron behavior, the time-dependent 
Boltzmann equation, is solved in a 7-dimensional phase space, ($3$ in space, $2$ 
in angle, and one each in energy and time). The scale of a nuclear reactor 
simulation is inherently large, spanning five orders of magnitude in space and 
ten in neutron energy. Resolved discretization across those scales would require 
over $10^{17}$ degrees of freedom per timestep, well beyond the
capabilities of even exascale computing. In this way, without sophisticated
data and analysis methods, we would run out of computational resources before
heat transport, fluid flow, or material performance in the reactor had even
been addressed. 

%correct a dearth of accuracy, transparency, and access in the current landscape of computational tools for nuclear engineering. 

\section*{\textcolor{gray}{\it Past and Current Efforts}}

The past and current work in this group has focused 
on the development of software tools seeking to tackle a variety of challenges 
in nuclear reactor physics. As such, the group has become a nexus of 
computational tool development for simulation environments and analysis toolkits 
in the field. These efforts are enabled by a vision that emphasizes adaptation 
of relevant data methodologies to domain-specific computational analysis. The 
following examples of current work demonstrate the way this vision is executed 
within this group.

\paragraph{Neutronics Methods on GPU Architectures}

One example of data-driven computational tool development in this research group 
is recent work in the area of neutronics methods for Graphical Processing Unit 
(GPU) computer architectures. This work, known as Weaving All the Random 
Particles (WARP), has sought to advance methods for monte carlo reactor physics 
analysis within a GPU framework. This work was acheived by applying modern 
algorithms for GPU monte carlo analysis to the specific structure of nuclear 
data and features reactor physics calculations.

\paragraph{Coupled Multiphysics for Advanced Reactor Analysis}

The postdoctoral research of Dr. Kathryn Huff is focused on the development of a 
high-fidelity, fully-coupled, multiphysics model of the Pebble-Bed, 
Fluoride-Salt-Cooled, High-Temperature Reactor (PB-FHR) 
\cite{facilitators_fluoride-salt-cooled_2013, 
facilitators_fluoride-salt-cooled_2013-1, 
facilitators_fluoride-salt-cooled_2013-2, 
facilitators_fluoride-salt-cooled_2013-3}. The objective of this work is to 
conduct high-fidelity simulations of transient event behavior in this complex 
reactor in a high-performance computing environment.

In her first months at Berkeley, she forged a collaboration with the Idaho 
National Laboratory to conduct this development within an open-source 
multiphysics framework, the Multiphysics Object Oriented Simulation Environment 
(MOOSE) \cite{gaston_moose:_2009}. This work will result in a transient, 
three-dimensional MOOSE module with fully coupled thermal-hydraulics and 
deterministic neutronics.  


\section*{\textcolor{gray}{\it Future Research}}

To this end, work in this group ranges 
from individual contributions to existing toolkits to comprehensive solution 
libraries for specific nuclear applications. Open source software projects such as 
MOOSE \cite{gaston_moose:_2009}, the Python for Nuclear Engineering (PyNE) 
toolkit \cite{pyne_pyne_2011}, and Cyclus \cite{huff_cyclus_2011} are fertile ground for contributing computational nuclear engineering tools with a wide impact 
potential. This research program will continue to leverage the significant effort represented 
by those tools, rather than reinventing the wheel, and will benefit from the 
user and developer communities that they possess.  Below, we will briefly expand 
on two near term focus areas which exemplify these goals.  

\paragraph{A Public Radiation Detection Database}

Members of the Reactor Design and Neutronics group are becoming involved in a 
nascent project called Kelp Watch \cite{vetter_kelp_2013}. This work revolves 
around the collection, analysis, and presentation of environmental radation 
detection data. 


\paragraph{Skills for Scientific Computing}
The Reactor Design and Neutronics group pursues advances in nuclear reator 
physics by means of simulation methods, 
efficient algorithms, novel computer architectures, and sophisticated design 
tailored to the many scales and multiple physics encountered in nuclear energy 
technology.  The tenets of scientific endeavor (e.g., data control, 
reproducibility, comprehensive documentation, and peer review) suffer in 
projects that fail to make use of current development tools such as unit 
testing, version control, automated documentation and others 
\cite{wilson_best_2014, merali_computational_2010}. 

Recently, structured education \cite{huff_hacker_2014, huff_rapid_2011} related 
to rigorous scientific computing standards has begun to help this research 
program avoid such pitfalls. 


\section*{\textcolor{gray}{\it Importance of BIDS}}

Incorporation of new numerical methods for scientific data management, 
exploration, and analysis will help to keep this research program relevant. 
It is our hope that advances in reactor physics will result from our 
domain-specific applications of the broadly applicable data science techniques, 
algorithms, and best practices that promise to be the focus of the BIDS 
collaboration.


Additionally, 

\bibliographystyle{unsrt}
\bibliography{participation}

\end{document}

