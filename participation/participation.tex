\documentclass[a4paper, 12pt]{article}
%\topmargin-1.5cm

\usepackage{fancyhdr}
\usepackage{pagecounting}
\usepackage[dvips]{color}

% Color Information from - http://www-h.eng.cam.ac.uk/help/tpl/textprocessing/latex_advanced/node13.html

% NEW COMMAND
% marginsize{left}{right}{top}{bottom}:
%\marginsize{3cm}{2cm}{1cm}{1cm}
%\marginsize{0.85in}{0.85in}{0.625in}{0.625in}

%\advance\oddsidemargin-0.85in
%\advance\evensidemargin-0.85in
%\textheight8.5in
%\textwidth6.75in
\newcommand\bb[1]{\mbox{\em #1}}
\def\baselinestretch{1.25}
%\pagestyle{empty}
\newcommand{\hsp}{\hspace*{\parindent}}
\definecolor{gray}{rgb}{0.4,0.4,0.4}

\newcommand{\authorname}{Reactor Design and Neutronics Group}
\newcommand{\longauthorname}{Dr. Denia~Djoki\'{c}\\Dr.  Kathryn~Huff\\Prof.~Rachel~Slaybaugh\\Prof.~Jasmina~Vuji\'{c} } 
\newcommand{\authorsite}{ucb-rdn.github.com}

\begin{document}
\pagestyle{fancy}
%\pagenumbering{gobble}
%\fancyhead[location]{text} 
% Leave Left and Right Header empty.
%\lhead{}
%\rhead{}
%\rhead{\thepage}
\lhead{\textcolor{gray}{\it \authorname}}
\rhead{\textcolor{gray}{\thepage/\totalpages{}}}
\renewcommand{\headrulewidth}{0pt} 
\renewcommand{\footrulewidth}{0pt} 
\fancyfoot[C]{\footnotesize \textcolor{gray}{\authorsite}} 

\begin{center}
{\LARGE \bf Participation}\\
\vspace*{0.1cm}
{\normalsize \longauthorname}
\end{center}
%\vspace*{0.2cm}

%\begin{document}
%\centerline {\Large \bf Research Statement for \authorname}
%\vspace{0.5cm}

% Write about research interests...
%\footnotemark
%\footnotetext{Check This}

% Your response to the “call for participation” should be a 2-3 page document 
% that describes your efforts in data science, including your challenges and 
% opportunities. We are not asking for a budget or a formal proposal. Instead, 
% please provide an overview of the scientific goals and vision and how working 
% within a lager data science context will further that vision. Our intent is 
% two-fold:  To build a community of data science, and help us steer and evolve 
% the mission of BIDS; and to help attract funding from other sources, including 
% new grants, support from private donors, and corporate sponsorships.

% Introduction
The Reactor Design and Neutronics group at the University of 
California--Berkeley \cite{reactor_design_and_neutronics_university_2013} is a team of professors, 
postdocs, and students who pursue advances in nuclear reactor physics by means of 
simulation methods, efficient algorithms, novel computer architectures, and 
sophisticated software design tailored to the many scales and multiple physics 
encountered in nuclear energy technology.  The direction of this group 
emphasizes the development of computational methods and simulation tools for 
nuclear reactor physics across many scales, with special focus on the safety and 
sustainability of advanced reactor designs and fuel cycles.  

It is our hope that novel solutions to these challenges in nuclear 
energy will arise from the development of application-driven numerical 
simulation methods and incorporation of sophisticated computational tools.  
We expect that involvement in the BIDS community will extend the 
scope of data and analysis methodologies within our wheelhouse and provide 
educational opportunities in software best practices to our students and 
researchers. Similarly, we expect that our approach to data-driven science 
promises to correspondingly expand horizons for the rest of the BIDS community.

\section*{\textcolor{gray}{\it Past and Current Efforts}}

The past and current work in this group has focused on the development of 
software toolkits and simulation environments tackling a variety of challenges 
in nuclear reactor physics. These efforts are enabled by a vision that 
emphasizes adaptation of relevant computational methods to domain-specific 
analysis. The following examples of current work demonstrate the 
execution of this vision.

\paragraph{Neutronics Methods on GPU Architectures}

One example of data-driven computational tool development in this research group 
is recent work in the area of neutronics methods for Graphical Processing Unit 
(GPU) computer architectures. This work, known as Weaving All the Random 
Particles (WARP), has sought to advance methods for Monte Carlo reactor physics 
analysis within a GPU framework. This work was achieved by applying modern 
algorithms for GPU Monte Carlo analysis to the specific structure of nuclear 
data and features reactor physics calculations.



\paragraph{Contributions to Open Source Toolkits}

Work in this group ranges from individual contributions in existing toolkits to 
comprehensive solution libraries for specific nuclear applications.  Open source 
software projects this group is involved in include the Python for Nuclear 
Engineering (PyNE) toolkit \cite{pyne_pyne_2011}, the Cyclus fuel cycle 
simulator \cite{huff_cyclus_2011}, the Serpent Monte Carlo neutronics code 
\cite{leppanen_serpentcontinuous-energy_2012}, and the Multiphysics Object Oriented Simulation 
Environment (MOOSE) \cite{gaston_moose:_2009}. These are fertile ground for 
contributing computational nuclear engineering tools with a wide impact 
potential and this research program will continue to leverage the significant 
effort represented by those and similar tools, rather than reinventing the 
wheel, and will benefit from the user and developer communities that they 
possess.  


\section*{\textcolor{gray}{\it Future Research}}

Below, we will briefly expand on three near term 
focus areas that are synergistic with the BIDS effort.  

\paragraph{Coupled Multiphysics for Advanced Reactor Analysis}

Work is beginning on a high-fidelity, fully-coupled, multiphysics model of the 
Pebble-Bed, Fluoride-Salt-Cooled, High-Temperature Reactor (PB-FHR) 
\cite{facilitators_fluoride-salt-cooled_2013, 
facilitators_fluoride-salt-cooled_2013-1, 
facilitators_fluoride-salt-cooled_2013-2, 
facilitators_fluoride-salt-cooled_2013-3}. The objective of this work is to 
conduct high-fidelity simulations of transient event behavior in this complex 
reactor in a high-performance computing environment.

This simulation will take place within an open-source 
multiphysics framework, MOOSE \cite{gaston_moose:_2009}. 
This framework utilizes the Jacobian Free 
Newton Krylov method for solving large systems of partial differential 
equations. 
Our domain-specific application of this work will result in a transient, 
three-dimensional MOOSE module with fully coupled thermal-hydraulics and 
deterministic neutronics. However, the MOOSE software is physics agnostic. Since 
it provides a powerful, broadly applicable simulation environment, it may prove 
relevant to a number of the scientific efforts involved in the BIDS effort. 


\paragraph{Teaching Skills for Scientific Computing}
The tenets of scientific endeavor (e.g., data control, 
reproducibility, comprehensive documentation, and peer review) suffer in 
projects that fail to make use of current development tools such as unit 
testing, version control, automated documentation, and others 
\cite{wilson_best_2014, merali_computational_2010}.  Recently, a peer-education 
initiative \cite{huff_hacker_2014} encouraging rigorous 
scientific computing standards and open science has begun to help the Reactor 
Design and Neutronics group to avoid such pitfalls.

While the current initiative takes the form of a seminar series targeted at 
graduate students, a recently submitted NSF proposal takes the form of a
short summer workshop targeted at undergraduates. That proposal, if funded, expects to 
benefit greatly from the presence of BIDS on campus by making use of expertise 
among the fellows and by engaging in the conversation that will be underway in 
the BIDS Education and Training working group.  


\paragraph{A Public Radiation Detection Database}

Members of the Reactor Design and Neutronics group are becoming involved in a 
nascent project called Kelp Watch \cite{vetter_about_2014}. This work revolves 
around the collection, analysis, and presentation of environmental radiation 
detection data. By collecting environmental radiation data and providing an API 
for exploration of that data, this project promises to provide an excellent 
resource for open science research on radiation and its effects on the 
environment. Just as importantly, though, is the curation, analysis, and 
visualization of that data for public consumption. 

\section*{\textcolor{gray}{\it Importance of BIDS}}

Incorporation of new numerical methods for scientific data management, 
exploration, and analysis will help to keep this research program relevant. 
It is our hope that advances in reactor physics will result from our 
domain-specific applications of the broadly applicable data science techniques, 
algorithms, and best practices that promise to be the focus of the BIDS 
collaboration.

Additionally, as we engage in scientific computing education and research, we 
hope to learn from and contribute to the conversations being held in the BIDS 
Working Groups on Education and Training, Software Tools and Environments, and 
Reproducible and Open Science.  

Finally, as our research group and our scientific field at large face the 
challenge of retaining the brightest minds, we look forward to guidance from the 
Working Group on Career Paths and Alternative Metrics.

\bibliographystyle{unsrt}
\bibliography{participation}

\end{document}

