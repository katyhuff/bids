\documentclass[a4paper, 12pt]{article}
%\topmargin-1.5cm

\usepackage{fancyhdr}
\usepackage{pagecounting}
\usepackage[dvips]{color}

% Color Information from - http://www-h.eng.cam.ac.uk/help/tpl/textprocessing/latex_advanced/node13.html

% NEW COMMAND
% marginsize{left}{right}{top}{bottom}:
%\marginsize{3cm}{2cm}{1cm}{1cm}
%\marginsize{0.85in}{0.85in}{0.625in}{0.625in}

%\advance\oddsidemargin-0.85in
%\advance\evensidemargin-0.85in
%\textheight8.5in
%\textwidth6.75in
\newcommand\bb[1]{\mbox{\em #1}}
\def\baselinestretch{1.25}
%\pagestyle{empty}
\newcommand{\hsp}{\hspace*{\parindent}}
\definecolor{gray}{rgb}{0.4,0.4,0.4}

\newcommand{\authorname}{Kathryn~D.~Huff }
\newcommand{\authoremail}{katyhuff@gmail.com}
\newcommand{\authorsite}{katyhuff.github.com}

\begin{document}
\pagestyle{fancy}
%\pagenumbering{gobble}
%\fancyhead[location]{text} 
% Leave Left and Right Header empty.
%\lhead{}
%\rhead{}
%\rhead{\thepage}
\lhead{\textcolor{gray}{\it \authorname}}
\rhead{\textcolor{gray}{\thepage/\totalpages{}}}
\renewcommand{\headrulewidth}{0pt} 
\renewcommand{\footrulewidth}{0pt} 
\fancyfoot[C]{\footnotesize \textcolor{gray}{\authorsite}} 

\begin{center}
{\LARGE \bf Funding}\\
\vspace*{0.1cm}
{\normalsize \authorname (\authoremail)}
\end{center}
%\vspace*{0.2cm}

%\begin{document}
%\centerline {\Large \bf Research Statement for \authorname}
%\vspace{0.5cm}

% Write about research interests...
%\footnotemark
%\footnotetext{Check This}

% Your response to the “call for participation” should be a 2-3 page document 
% that describes your efforts in data science, including your challenges and 
% opportunities. We are not asking for a budget or a formal proposal. Instead, 
% please provide an overview of the scientific goals and vision and how working 
% within a lager data science context will further that vision. Our intent is 
% two-fold:  To build a community of data science, and help us steer and evolve 
% the mission of BIDS; and to help attract funding from other sources, including 
% new grants, support from private donors, and corporate sponsorships.

% Introduction

\section*{\textcolor{gray}{\it Current Funding}}
My postdoctoral appointment is currently funded at \$60,000 annually. I have a 
two year appointment which began in September of 2013.
My funding comes equally from two sources. The first is the Nuclear Science and Security 
Consortium (NSSC), in which I am a fellow. The second is a DOE-NE Integrated 
Research Project grant on Fluoride-Salt-Cooled High-Temperature Reactors (FHRs).  

\section*{\textcolor{gray}{\it Pending Funding}}
Three current funding proposals are currently pending in relation to my work in 
scientific computation and education.

First, a DOE NEUP pre-proposal has been submitted for which I am the principle
investigator. This proposal seeks to contribute key functionality to an open source
nuclear fuel cycle simulation tool, Cyclus, and to conduct novel analysis
investigating the potential impacts of a used nuclear fuel recycling scheme. 

Additionally, I am an author and listed participant in a recently submitted 
multi-institution NSF grant proposal seeking to support scientific computing 
bootcamps targeted at undergraduates. This proposal was the beneficiary of 
letters of commitment from the BIDS team and D-Lab teams at Berkeley. This work, 
if awarded, will emphasize assessment of boot camp effectiveness and will 
ideally engage closely with the related educational interests and activities at 
D-Lab.

Finally, on behalf of the PyNE toolkit team, I have submitted a Google Summer of 
Code application to support student contributors in the summer of 2014.

\section*{\textcolor{gray}{\it Potential Follow-On Funding }}
%What direction will your research take you in next
%what new questions do you have?
% external research funding

To the extent possible, I will seek funding for three areas of need within my 
research program. First, I will seek funding for my own time and effort working 
on contributions to open source nuclear data toolkits and high fidelity 
multiphysics simulations of nuclear reactor accident transients. Second, I will 
seek funding for computing resources, where necessary. Finally, I will seek 
funding for graduate and undergraduate students whom I would enjoy mentoring in 
their contributions to open source nuclear reactor physics.

My future research plans fit well into the Department of Energy (DOE) Nuclear 
Energy University Programs (NEUP) funding scope. In particular, both Cyclus and 
MOOSE applications have been specifically supported in recent workscopes, 
indicating that these are areas of sustained interest within the DOE-NE. 
Since Cyclus' recent adoption by the DOE Office of Nuclear Energy (DOE-NE)  as 
its flagship nuclear fuel cycle simulator, it has garnered an international, 
multi-institution collaboration of users and developers including more than 
three NEUP-funded projects. 

Nonproliferation applications of fuel cycle analysis have the potential to 
benefit greatly from funding opportunities through the National Nuclear Security 
Administration (NNSA), Nuclear Regulatory Commission (NRC), and Department of 
Homeland Security Domestic Nuclear Detection Office (DHS-DNDO). These pursuits 
in combination with my focus on safety may strengthen any applications I may 
make in the next stages of my career to junior faculty development awards 
available from the National Science Foundation (NSF), DOE, DHS-DNDO, the Office 
of Naval Research Research (ONR), and the Air Force Office of Scientific 
Research (AF-OSR).

Finally, I also expect to continue to seek additional support for methods development and 
contribution to computational nuclear toolkits from programs such as Google 
Summer of Code and NumFocus. Furthermore, collaboration with Nuclear Engineering Advanced 
Modeling and Simulation campaign with current and past colleagues at Idaho, 
Argonne, and Oak Ridge National Laboratories may provide additional student and 
collaboration support in the area of modeling and simulation. 


\bibliographystyle{unsrt}
\bibliography{funding}

\end{document}

