\documentclass[a4paper, 10pt]{article}
%\topmargin-1.5cm

\usepackage{fancyhdr}
\usepackage{pagecounting}
\usepackage[dvips]{color}

% Color Information from - http://www-h.eng.cam.ac.uk/help/tpl/textprocessing/latex_advanced/node13.html

% NEW COMMAND
% marginsize{left}{right}{top}{bottom}:
%\marginsize{3cm}{2cm}{1cm}{1cm}
%\marginsize{0.85in}{0.85in}{0.625in}{0.625in}

%\advance\oddsidemargin-0.85in
%\advance\evensidemargin-0.85in
%\textheight8.5in
%\textwidth6.75in
\newcommand\bb[1]{\mbox{\em #1}}
\def\baselinestretch{1.25}
%\pagestyle{empty}
\newcommand{\hsp}{\hspace*{\parindent}}
\definecolor{gray}{rgb}{0.4,0.4,0.4}

\newcommand{\authorname}{Kathryn~D.~Huff }
\newcommand{\authoremail}{katyhuff@gmail.com}
\newcommand{\authorsite}{katyhuff.github.com}

\begin{document}
\pagestyle{fancy}
%\pagenumbering{gobble}
%\fancyhead[location]{text} 
% Leave Left and Right Header empty.
%\lhead{}
%\rhead{}
%\rhead{\thepage}
\lhead{\textcolor{gray}{\it \authorname}}
\rhead{\textcolor{gray}{\thepage/\totalpages{}}}
\renewcommand{\headrulewidth}{0pt} 
\renewcommand{\footrulewidth}{0pt} 
\fancyfoot[C]{\footnotesize \textcolor{gray}{\authorsite}} 

\begin{center}
{\LARGE \bf Funding}\\
\vspace*{0.1cm}
{\normalsize \authorname (\authoremail)}
\end{center}
%\vspace*{0.2cm}

%\begin{document}
%\centerline {\Large \bf Research Statement for \authorname}
%\vspace{0.5cm}

% Write about research interests...
%\footnotemark
%\footnotetext{Check This}

% Your response to the “call for participation” should be a 2-3 page document 
% that describes your efforts in data science, including your challenges and 
% opportunities. We are not asking for a budget or a formal proposal. Instead, 
% please provide an overview of the scientific goals and vision and how working 
% within a lager data science context will further that vision. Our intent is 
% two-fold:  To build a community of data science, and help us steer and evolve 
% the mission of BIDS; and to help attract funding from other sources, including 
% new grants, support from private donors, and corporate sponsorships.

% Existing extramural support and plans for generating follow on funding

% Introduction

\section*{\textcolor{gray}{\it Current Funding}}
My postdoctoral appointment is currently funded at \$60,000 annually. I have a 
two year appointment which began in September of 2013.  My funding comes equally 
from two sources. The first is the Nuclear Science and Security Consortium 
(NSSC), in which I am a fellow. The second is a DOE-NE Integrated Research 
Project grant on Fluoride-Salt-Cooled High-Temperature Reactors (FHRs).  

\section*{\textcolor{gray}{\it Pending Funding}}
Three proposals are currently pending in relation to my work in scientific 
computation and education.

First, a Department of Energy (DOE) Nuclear Energy University Programs (NEUP) 
pre-proposal has been submitted for which I am the principle investigator. It 
has been favorably reviewed and was invited for submission as a full proposal. 
This proposal seeks to contribute key functionality to an open source nuclear 
fuel cycle simulation tool, Cyclus, and to conduct novel analysis investigating 
the potential impacts of a used nuclear fuel recycling scheme. It would fund a 
graduate student full time, and would fund me as a mentor at 10\%. 

Additionally, I am an author and listed participant in a recently submitted 
multi-institution National Science Foundation (NSF) grant proposal seeking to 
support scientific computing bootcamps targeted at undergraduates. This proposal 
was the beneficiary of letters of commitment from the BIDS team and D-Lab teams 
at Berkeley. This work, if awarded, will emphasize assessment of boot camp 
effectiveness and will ideally engage closely with the related educational 
interests and activities at the D-Lab.

Finally, on behalf of the PyNE toolkit team, I am pursuing private foundation 
funding to support student contributors in the summer of 2014.

\section*{\textcolor{gray}{\it Potential Follow-On Funding }}
%What direction will your research take you in next
%what new questions do you have?
% external research funding

To the extent possible, I will continue to seek funding for three areas of need 
within my research program. First, I will seek funding for my own time and 
effort working on contributions to open source nuclear data toolkits and high 
fidelity multiphysics simulations of nuclear reactor accident transients. 
Second, I will seek funding for computing resources, where necessary. Finally, I 
will seek support for graduate and undergraduate students whom I would enjoy 
mentoring toward contributions to open source nuclear reactor physics. 

My nuclear science focused research plans fit well into the funding scope of the 
Department of Energy (DOE) Nuclear Energy University Programs (NEUP), the 
National Nuclear Security Administration (NNSA), Nuclear Regulatory Commission 
(NRC), and Department of Homeland Security Domestic Nuclear Detection Office 
(DHS-DNDO).  

However, traditional funding pathways for reactor physics rarely include 
specific support for computational methods development, or mentoring student 
contributions to open source nuclear science and engineering software toolkits. 
I hope that an affiliation with BIDS could open pathways to funding sources with 
support for those computational methods development pursuits and mentoring opportunities. 


\end{document}

