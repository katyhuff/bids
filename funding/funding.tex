\documentclass[a4paper, 12pt]{article}
%\topmargin-1.5cm

\usepackage{fancyhdr}
\usepackage{pagecounting}
\usepackage[dvips]{color}

% Color Information from - http://www-h.eng.cam.ac.uk/help/tpl/textprocessing/latex_advanced/node13.html

% NEW COMMAND
% marginsize{left}{right}{top}{bottom}:
%\marginsize{3cm}{2cm}{1cm}{1cm}
%\marginsize{0.85in}{0.85in}{0.625in}{0.625in}

%\advance\oddsidemargin-0.85in
%\advance\evensidemargin-0.85in
%\textheight8.5in
%\textwidth6.75in
\newcommand\bb[1]{\mbox{\em #1}}
\def\baselinestretch{1.25}
%\pagestyle{empty}
\newcommand{\hsp}{\hspace*{\parindent}}
\definecolor{gray}{rgb}{0.4,0.4,0.4}

\newcommand{\authorname}{Kathryn~D.~Huff }
\newcommand{\authoremail}{katyhuff@gmail.com}
\newcommand{\authorsite}{katyhuff.github.com}

\begin{document}
\pagestyle{fancy}
%\pagenumbering{gobble}
%\fancyhead[location]{text} 
% Leave Left and Right Header empty.
%\lhead{}
%\rhead{}
%\rhead{\thepage}
\lhead{\textcolor{gray}{\it \authorname}}
\rhead{\textcolor{gray}{\thepage/\totalpages{}}}
\renewcommand{\headrulewidth}{0pt} 
\renewcommand{\footrulewidth}{0pt} 
\fancyfoot[C]{\footnotesize \textcolor{gray}{\authorsite}} 

\begin{center}
{\LARGE \bf Participation}\\
\vspace*{0.1cm}
{\normalsize \authorname (\authoremail)}
\end{center}
%\vspace*{0.2cm}

%\begin{document}
%\centerline {\Large \bf Research Statement for \authorname}
%\vspace{0.5cm}

% Write about research interests...
%\footnotemark
%\footnotetext{Check This}

% Your response to the “call for participation” should be a 2-3 page document 
% that describes your efforts in data science, including your challenges and 
% opportunities. We are not asking for a budget or a formal proposal. Instead, 
% please provide an overview of the scientific goals and vision and how working 
% within a lager data science context will further that vision. Our intent is 
% two-fold:  To build a community of data science, and help us steer and evolve 
% the mission of BIDS; and to help attract funding from other sources, including 
% new grants, support from private donors, and corporate sponsorships.

% Introduction

\section*{\textcolor{gray}{\it Current Funding}}
My current funding comes from ($50\%$) the Nuclear Science and Security 
Consortium (NSSC) and ($50\%$) from a DOE-NE Integrated Research Projects on 
Fluoride-Salt-Cooled High-Temperature Reactors (FHRs).  

\section*{\textcolor{gray}{\it Pending Funding}}
A DOE NEUP pre-proposal has been submitted, for which I am the principle
investigator, which seeks to contribute key functionality to an open source
nuclear fuel cycle simulation tool, Cyclus, and to conduct novel analysis
investigating the potential impacts of a used nuclear fuel recycling scheme. 

In accordance with its mission, the Fuel Cycle Technologies Program seeks to
enable advances in fuel technology via modeling and simulation of fuel cycle
options [Price2013]. Large impacts on LWR closed fuel cycles and waste
management may result from advances in laser enrichment technology for
plutonium isotopic separation. The safety, economic, and environmental impacts
of this disruptive technology have yet to be studied in detail within the
context of fuel cycle options. The proposed work will fill this knowledge gap.

Historically fuel cycles have been defined by two types of separations:
chemical separations and uranium isotopic separations. Plutonium isotopic
separation creates a third dimension that may allow infinite recycle of
plutonium in LWRs and reduce the cost of the geological repository.  

The primary focus of the proposed work will conduct comparative analyses of the
technical implications plutonium isotopic separations schemes within canonical
open and closed fuel cycle scenarios modeled by the Cyclus fuel cycle simulator
[Wilson2012]. As a part of this effort, separations process algorithms,
supporting material data, and metrics calculation utilities will be contributed
to the developer toolkit available to the Cyclus ecosystem. 

The Cyclus simulation framework is incomplete without a versatile ecosystem of
models to represent innovative technologies and processing strategies. This
work will leverage the Cyclus computational methodology and software toolkit
toward fuel cycle simulations that investigate the fuel cycle implications of
plutonium isotopic separation. Additionally, by contributing this isotopic
separations Facility Model to the ecosystem of available models, the
implications of this and other innovative isotopic separations technologies on
the broader fuel cycle can be captured. 

Emphasis on the ease of use of these models will seed collaboration with
researchers interested in modeling additional technology and policy impacts of
other innovative isotopic separation concepts. For instance, this capability
could greatly assist the Fuel Cycle Technologies Program in the long term as a
tool for focusing R&D among the array of separations options available to the
Fuel Cycle Technologies mission.



\section*{\textcolor{gray}{\it Potential Follow-On Funding }}
%What direction will your research take you in next
%what new questions do you have?
% external research funding
Novel solutions to challenges in nuclear energy such as nuclear reactor accident 
response, nuclear fuel cycle strategy, and waste management will arise from the  
development of application-driven numerical simulation methods and incorporation 
of sophisticated computational tools. To this end, my work will range from 
individual contributions to existing toolkits to comprehensive solution 
libraries for specific nuclear applications (e.g., a suite of activation 
analysis methods for sub-critical systems). Open source projects such as 
Cyclus \cite{huff_cyclus_2011}, MOOSE \cite{gaston_moose:_2009}, and the Python 
for Nuclear Engineering (PyNE) toolkit \cite{pyne_pyne_2011} are fertile ground 
for contributing computational nuclear engineering tools with a wide impact 
potential. My research program will leverage the significant effort represented 
by those tools, rather than reinventing the wheel, and will benefit from the 
user and developer communities that they possess.  I will briefly expand on two 
of my near term focus areas which exemplify these goals. 

\paragraph{Advanced Technology Modeling for Fuel Cycle Analysis} 

One of the continuing challenges in evaluating the system level impacts of 
nuclear technologies and policies is the need for a standardized simulation 
platform and library of models representing nuclear technologies and calculating 
fuel cycle metrics.  The Cyclus simulation framework 
has provided this standardizing platform, but it is incomplete without a 
robust library of fuel cycle metric calculation methods and a versatile ecosystem of 
models to represent innovative nuclear technologies. My work in this area will 
fill this gap. 

The Cyclus modeling paradigm enables both scientific and policy analyses by 
following transactions of discrete quanta of material among discrete facilities, 
arranged in a geographic and institutional framework, and trading in flexible 
markets. Cyclus' sophisticated design emphasizes separation of the core simulation logic 
from the technical nuclear process models representing the facilities of the 
nuclear fuel cycle. This ensures robust modularity with regard to functionality.  

My work will leverage Cyclus toward the development of a library 
of methods for calculating fuel cycle metrics.  Object-oriented methods enabled 
by the unique Cyclus modeling paradigm will result in metrics with richer detail 
than has historically been possible with previous simulators. One example is a 
``shadow fuel cycle'' nonproliferation calculation strategy in which nuclear material is 
diverted by a nefarious actor within a facility to determine the minimum 
detectable diverted material amount.  This metric is used to describe the 
detection sensitivity of inspections.  Creative safeguards strategies may emerge 
from the richness of disaggregated detail made possible by evaluating this 
nefarious theft scenario within the context of the discrete materials and 
discrete facilities of the Cyclus environment.

Additionally, my work will include both focused, single technology model development and 
development of concise parametric models capable of representing myriad 
technologies.  Such models will enable assessment of a broad range of technology 
and policy implications related to the introduction of those technologies into 
the world nuclear energy market. By balancing speed with the capability to 
capture dominant physics, Cyclus can rapidly assess the implications of those 
technologies on the broader fuel cycle.  Contributing advanced reactor and fuel 
cycle concepts to the ecosystem of available models will also increase the 
potential for collaboration with other domestic and international researchers in 
technology evaluations and community benchmarking exercises.

I envision, for example, advising a student interested in innovative reactor 
design to develop a parameterized spectral model to approximate burnup and 
transmutation physics with appropriate speed and fidelity for fuel cycle 
simulations.  This work will benefit from my experience navigating this 
trade-off between speed and fidelity, in which a model must be rapid while 
simultaneously detailed enough to uncover system level responses to technology 
choices.


\paragraph{Multiphysics Model Development for Reactor Design and Analysis}

Another focus of my research program will include the development of 
multiphysics models of reactor designs, with particular focus on those boasting 
inherent safety features (i.e. accident tolerant fuels or non-voiding coolants).  
Design- and Beyond-Design-Basis Accident simulations are essential to the 
advancement of nuclear reactor safety. Faithful assessments of reactor response 
in these scenarios require fully coupled, transient simulation of neutronics, 
thermal hydraulics, and structural performance, necessitating specialized 
computational methodologies. 

In particular, the Jacobian-Free Newton-Krylov (JFNK) solution method 
\cite{knoll_jacobian-free_2004}, combined with 
physics-based preconditioning, enables extraordinary parallel scalability and 
and fully-coupled solutions to systems of neutron transport and thermal 
hydraulic equations. MOOSE, from INL, which relies on this exceptional numerical 
method as well as adaptive mesh refinement for structured and 
unstructured meshes, is beginning to be used in earnest for nuclear engineering 
applications (e.g., \cite{park_tightly_2009, short_multiphysics_2013, 
novascone_assessment_2012, novascone_multidimensional_2012, 
gaston_parallel_2009} among others). Moreover, the MOOSE tool possesses a 
modular, object-oriented simulation environment approach, similar to that of 
Cyclus, which allows user-developers to construct applications by focusing on 
the unique physics of their modeling problem  with minimal concern for the 
system solution methodology. 

My future work with MOOSE, continuing a recent collaboration that I forged with the 
MOOSE team while at UC-Berkeley, will focus on the development 
of MOOSE `Applications' capable of 3-dimensional, multi-scale, massively parallel 
analyses of promising reactor technologies. This will consist of developing 
dimension-agnostic, application-driven physics 
`Kernels,' combining them with validated kernels developed by others, and 
constructing them into coherent simulation objects. 

In particular, I am interested in the potential impacts of the fully coupled 
nature of MOOSE multiphysics. The ability to eliminate or nearly eliminate the 
scaling and coupling distortions seen in simulations that are only loosely or 
tightly coupled is a breakthrough capability that could change the face of 
reactor modeling entirely.  

\paragraph{External Research Funding}
My future research plans fit well into the Department of Energy (DOE) Nuclear 
Energy University Programs (NEUP) funding scope. In particular, both Cyclus and 
MOOSE applications have been supported in recent workscopes. Since Cyclus' 
recent adoption by the DOE Office of Nuclear Energy (DOE-NE)  as its flagship 
nuclear fuel cycle simulator, it has garnered an international, 
multi-institution collaboration of users and developers including more than 
three NEUP-funded projects. 

The inclusion of the Cyclus project in previous workscopes indicates that 
nuclear fuel energy system simulation development with the Cyclus tool is an 
area of sustained interest within the DOE-NE.  I have already submitted, as 
principal investigator, a 2014 NEUP pre-proposal to support work on advanced 
laser separation technology modeling for fuel cycle analysis.  I expect that my 
intimate familiarity with Cyclus will add strength to this and future proposals.  

Nonproliferation applications of fuel cycle analysis have the potential to benefit greatly from funding 
opportunities through the National Nuclear Security Administration (NNSA), 
Nuclear Regulatory Commission (NRC), and Department of Homeland Security 
Domestic Nuclear Detection Office (DHS-DNDO). These pursuits in combination with 
my focus on safety may  strengthen my applications to junior faculty development 
awards available from the National Science Foundation (NSF), DOE, DHS-DNDO, the 
Office of Naval Research Research (ONR), and the Air Force Office of Scientific 
Research (AF-OSR).

Finally, I also expect to seek additional support for methods development and 
contribution to computational nuclear toolkits from programs such as Google 
Summer of Code, the Alfred P. Sloan Foundation, and the Gordon and Betty Moore 
Foundation.  The NSF is also interested in 
scientific computing and I am currently assisting in the collaborative preparation 
of a proposal for funding from NSF in the field of scientific computing 
education. Furthermore, collaboration with Nuclear Engineering Advanced 
Modeling and Simulation campaign with current and past colleagues at Idaho, 
Argonne, and Oak Ridge National Laboratories may provide additional student and 
collaboration support in the area of modeling and simulation. 

I have had many opportunities to participate in both failed and successful grant 
proposals.  As a graduate student, I assisted in writing grant proposals with my 
advisor, Paul P.H. Wilson. In 2012, one of those reached success, securing a 
three year, \$1.2 million, Nuclear Energy University Programs (NEUP) Research 
and Development grant for the expansion of Cyclus. Additionally, I was the 
primary author on an NEUP proposal for the 2013 call for proposals. I was 
pleased to have been invited to submit a full proposal, but that full proposal 
did not progress further. It is my hope that these early experiences with grant 
writing will serve me well in my own research program.




\bibliographystyle{unsrt}
\bibliography{participation}

\end{document}

