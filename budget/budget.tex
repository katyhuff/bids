\documentclass[a4paper, 10pt]{article}
%\topmargin-1.5cm

\usepackage{fancyhdr}
\usepackage{pagecounting}
\usepackage[dvips]{color}

% Color Information from - http://www-h.eng.cam.ac.uk/help/tpl/textprocessing/latex_advanced/node13.html

% NEW COMMAND
% marginsize{left}{right}{top}{bottom}:
%\marginsize{3cm}{2cm}{1cm}{1cm}
%\marginsize{0.85in}{0.85in}{0.625in}{0.625in}

%\advance\oddsidemargin-0.85in
%\advance\evensidemargin-0.85in
%\textheight8.5in
%\textwidth6.75in
\newcommand\bb[1]{\mbox{\em #1}}
\def\baselinestretch{1.25}
%\pagestyle{empty}
\newcommand{\hsp}{\hspace*{\parindent}}
\definecolor{gray}{rgb}{0.4,0.4,0.4}

\newcommand{\authorname}{Kathryn~D.~Huff }
\newcommand{\authoremail}{katyhuff@gmail.com}
\newcommand{\authorsite}{katyhuff.github.com}

\begin{document}
\pagestyle{fancy}
%\pagenumbering{gobble}
%\fancyhead[location]{text} 
% Leave Left and Right Header empty.
%\lhead{}
%\rhead{}
%\rhead{\thepage}
\lhead{\textcolor{gray}{\it \authorname}}
\rhead{\textcolor{gray}{\thepage/\totalpages{}}}
\renewcommand{\headrulewidth}{0pt} 
\renewcommand{\footrulewidth}{0pt} 
\fancyfoot[C]{\footnotesize \textcolor{gray}{\authorsite}} 

\begin{center}
{\LARGE \bf Budget}\\
\vspace*{0.1cm}
{\normalsize \authorname (\authoremail)}
\end{center}
%\vspace*{0.2cm}

%\begin{document}
%\centerline {\Large \bf Research Statement for \authorname}
%\vspace{0.5cm}

% Write about research interests...
%\footnotemark
%\footnotetext{Check This}

% Your response to the “call for participation” should be a 2-3 page document 
% that describes your efforts in data science, including your challenges and 
% opportunities. We are not asking for a budget or a formal proposal. Instead, 
% please provide an overview of the scientific goals and vision and how working 
% within a lager data science context will further that vision. Our intent is 
% two-fold:  To build a community of data science, and help us steer and evolve 
% the mission of BIDS; and to help attract funding from other sources, including 
% new grants, support from private donors, and corporate sponsorships.

% Existing extramural support and plans for generating follow on funding

% Introduction

\begin{table}[h!]
  \centering
  \begin{tabular}{|l|r|r|r|r|}
    \hline
    \textbf{Item} & \textbf{BIDS } & \textbf{NSSC} & \textbf{FHR} & \textbf{Total}\\
    \hline
    Annual Salary  & 30000 & 15000 & 15000 & 60000\\
    Overhead (58\%)  & 17400 & 8700 & 8700 & 34800\\
    Travel, SciPy & 2000 & 0 & 0 & 2000\\
    Other Travel & 0 & 4000 & 0 & 4000\\
    Total & 49400 & 27700 & 23700 & 100800\\
    \hline
  \end{tabular}
  \caption{Estimated budget for a single academic year. This budget would apply 
    to each of the 2014-2015 and 2015-2016 academic years.}
  \label{tab:budget}
\end{table}

\section*{\textcolor{gray}{\it Budget Justification}}
My postdoctoral appointment is currently funded at \$60,000 annually. I have a
two year appointment which is two academic years long and began in September of
2013.  My funding comes equally from two sources. The first is the Nuclear
Science and Security Consortium (NSSC), in which I am a fellow. The second is a
DOE-NE Integrated Research Project grant on Fluoride-Salt-Cooled
High-Temperature Reactors (FHRs).  

NSSC also funds necessary travel associated with my nuclear science research,
but cannot fund travel unrelated to nulear science such as the Scientific
Computing with Python Conference, SciPy. I have been a Technical
Program Chair for the last two years and this activity is a major feature of my
volunteer work in open science, scientific computing education, and
reproducible research. Travel to SciPy would be an essential activity for my
BIDS-related research direction.

If I were awarded 50\% funding for two academic years (September 2014 --
September 2016) through the BIDS Data Science Fellowship, NSSC and the FHR
project would correspondingly provide support for me at the 50\% level through
September 2016.

\end{document}

